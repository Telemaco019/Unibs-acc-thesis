\section{Bitcoin anonymity issues} In bitcoin the transactions are exchanged
between addresses, which, as explained in the previous chapters, are basically
hashes of public keys. The purpose of these addresses is to serve as pseudonyms
and provide some anonymity. However, since all Bitcoin transactions are stored
in a publicly available ledger and they basically consist of a chain of
digital signatures which provide cryptographic proofs of funds transfer, Bitcoin
privacy concerns were raised and in the last few years, researchers have shown
that Bitcoin anonymity is much weaker than was initially expected. Users’
transactions in fact can often be easily linked together and if anyone of those
transactions is linked to the user’s identity, then all of its transactions may
be exposed. For these reasons Bitcoin is sometimes compared to a bank making
bank statements publicly available online but blanking out the names.

\paragraph{Tainted bitcoins} The main consequence of this lack of anonymity is
that the history of each bitcoin can be traced and therefore some funds (for example
stolen bitcoins or bitcoins known to have been used for illegal purposes) can be
marked as \emph{tainted}, deflating consequently their value.  This can be done
for example by warning users to don't accept coins that come from a given
address using alert messages or by coding a list of banned Bitcoin addresses
within the official Bitcoin client releases. Also, users are less likely to
accept coins that are tainted since owning them  can be a risk and they can ask
the payer to use different coins as payment.


\paragraph{Clients privacy measures} Besides addresses, Bitcoin clients
adopt some more privacy measures. These measures consist of allowing users to
have more than one address and encouraging them to frequently change their
addresses by transferring some of their bitcoins to the newly created addresses.
Moreover, for each user, a new address is automatically created and used for
collecting the change resulting from any transaction of the user. These
addresses are called \emph{shadow addresses}.







%%%%%%%%%%%%%%%%%%%%%%%%%%%
% *** FIRST SUB-SECTION ***
%%%%%%%%%%%%%%%%%%%%%%%%%%%
\subsection{Compromise of privacy examples} In the following section, it will be
shown some examples of how user privacy can be compromised by exploiting the
existing Bitcoin client implementations and carrying out a behavior-based
analysis of the public ledger \cite{karame2016bitcoin}. It's important to point
out that there are also other kinds of attacks which operates at the network
layer and which allow the attacker to obtain user information from the Bitcoin
peer to peer network \cite{karame2016bitcoin}, which however will not be
discussed in this book.

\paragraph{Exploiting multi input transactions} The first method for obtaining
users information consists of observing the multi-input transactions. A
multi-input transaction, as discussed in chapter \ref{sec:transactions}, is a
transaction which accepts more UXTO as input. If in a transaction these UXTOs
are owned by different addresses, then it is straightforward to conclude that
the input addresses belong to the same user.

\paragraph{Exploiting shadow addresses} Another method is to exploit the shadow
addresses generated by the Bitcoin clients. When a Bitcoin transaction has $n$
output addresses $\{a_1 , \dots, a_n \}$ (transaction with multiple recipients)
and only one address is a new address (namely the address has never appeared in
the ledger before) it is then possible to assume that the newly appearing
address is a shadow address for the user that sent the transaction.

\paragraph{Behavior-based analysis} Besides exploiting Bitcoin client
implementations, an attacker can also use behavior-based clustering algorithms
like K-Means (KMC) and Hierarchical Agglomerative Clustering (HAC) for profiling
Bitcoin users. Without going into the details of these techniques, in reference
\ref{} these algorithms has been tested in a simulated Bitcoin system and the
achieved results shows that given 200 simulated user profiles, almost 42\% of
the users have their profiles captured with 80\% accuracy and the profile
leakage in Bitcoin is larger when users participate in a large number of
transactions, while decreases as the number of transactions performed by the
user decreases.
