\section{Bitcoin anonimity issues} In bitcoin the transactions are exchanged
between adresses, which, as explained in  the previous chapters are basically
hashes of public keys. The purpose of these adresses is to serve as pseudonyms
and provide some anonymity. However, since all Bitcoin transactions are stored
in a publicly available ledger and transactions basically consist of a chain of
digital signatures which provide cryptographic proofs of funds transfer, Bitcoin
privacy concerns were raised and in the last few years researchers have shown
that Bitcoin anonymity is much weaker than was initially expected. Users’
transactions in fact can often be easily linked together and if any one of those
transactions is linked to the user’s identity, then all of its transactions may
be exposed.


\paragraph{Tainted bitcoins} In Bitcoin terminology, tainting means measuring
the correlation between two (wallet) addresses
\cite{karame2016bitcoin},\cite{guthm_2016}. Since the ledger is publicly
available, any entity can therefore taint coins that belong to a specific set of
addresses and monitor their expenditure across the network. Coin tainting could
be used for example for allowing users to decide to stop interacting with
adresses that misbehave, deflating consequently the value of all the coins
pertaining to those address.



Bitcoin transactions basically consist of a chain
of digital signatures, which makes it possible to track the expenditure of
individual bitcoins. This allows any entity to taint bitcoins that belong to a
specific set of addresses and monitor their expenditure across the network.




%%%%%%%%%%%%%%%%%%%%%%%%%%%
% *** FIRST SUB-SECTION ***
%%%%%%%%%%%%%%%%%%%%%%%%%%%
\subsection{Bitcoin clients privacy measures} Besides address, Bitcoin clients
adopts some more privacy measures. These measures consist of allowing users to
have more than one address and encouraging them to frequently change their
adresses by transferring some of their bitcoins to the newly created addresses.
Moreover, for each user a new address is automatically created and used for
collecting the change resulting from any transaction of the user. These
addresses are called \emph{shadow addresses}.






%%%%%%%%%%%%%%%%%%%%%%%%%%%
% *** SECOND SUB-SECTION ***
%%%%%%%%%%%%%%%%%%%%%%%%%%%
\subsection{Compromise of privacy examples} In the following section it will be
shown some examples of how user privacy can be compromised by exploiting the
existing Bitcoin client implementations and carrying out a behaviour-based
analysis of the public ledger \cite{karame2016bitcoin}. It's important to point
out that there are also other kind of attacks which operates at the network
layer and which allow the attacker to obtain user information from the Bitcoin
peer to peer network \cite{karame2016bitcoin}, which however will not be
discussed in this book.

\paragraph{Exploiting multi input transactions} The first method for obtain
users information consists of observing the multi-input transactions. A
multi-input transaction, as discussed in chapter \ref{sec:transactions}, is a
transaction which accept  more UXTO as input. If in a transaction these UXTOs
are owned by different addresses, then it is straightforward to conclude that
the input addresses belong to the same user.

\paragraph{Exploiting shadow addresses} Another method is to exploit the shadow
addresses generated by the Bitcoin clients. When a Bitcoin transaction has $n$
output addresses $\{a_1 , \dots, a_n \}$ (transaction with multiple recipients)
and only one address is a new address (namely the address has never appeared in
the ledger before) it is then possible to assume that the newly appearing
address is a shadow address for the user that sent the transaction.

\paragraph{Behavior-based analysis} Besides exploiting Bitcoin client
implementations, an attacker can also use behavior-based clustering algorithms
like K-Means (KMC) and Hierarchical Agglomerative Clustering (HAC) for profiling
Bitcoin users. Without going into the datails of these techniques, in reference
\ref{} these algorithms has been tested in a simulated Bitcoin system and the
achieved results shows that given 200 simulated user profiles, almost 42\% of
the users have their profiles captured with 80\% accuracy and the profile
leakage in Bitcoin is larger when users participate in a large number of
transactions, while decreases as the number of transactions performed by the
user decreases.
