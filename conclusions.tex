\section{Conclusions}
Blockchain is an increasingly popular technology. However, despite the hype around it that lately manifested, it does not represent the solution to every problem. Indeed, as well as being a relatively young and immature technology, Blockchain generally well fits only to a specific set of problems characterized by certain properties. 
In particular, Blockchain turns out to be useful in all those problems in which a shared database is required, the data is updated frequently (it's not static data), more different entities read and write to the database and the history of the data has to be auditable. Blockchain, given its distributed nature, is also particularly useful if the involved parts do not trust each others and a central trusted authority cannot be used for controlling the data. On the other hand, Blockchain usually does not represent a good solution in those problems in which performance are foundamental, the involved data is strictly confidential and the rules of the transactions change frequently. In conclusions, the utility of Blockchain strongly depends on the problem context and on its application and, when evaluating whether to use Blockchain, one should generally consider three main aspects: 


\begin{itemize}
    \item the number of participants of the system
    \item the geographical distribution of the participants
    \item the performance requirements of the system
\end{itemize}

The most popular application of Blockchain are cryptocurrencies. The first and perhaps most famous one is Bitcoin, which has been the first real application of the Blockchain technology. Compared to the traditional currencies, Bitcoin has several advantages, mainly due to its distributed nature resulting from Blockchain adoption. The main advantages are the absence of central authorities, transparency, security and better anonymity. As consequence of these characteristics, Bitcoin gained a lot of popularity in the last few years, raising thus concerns about its ability to scale up in order to support the increasingly number of users. These concerns turn out to be founded since, as discussed in chapter \ref{sec:scalability}, the Bitcoin system currently has several bottlenecks and limitations that should be addressed in order to meet the scalability requirements.

Besides the scalability issues, Bitcoin privacy has been proved to be not as strong as expected. In particular, as discussed in chapter \ref{sec:privacy-enhancing}, by looking at the Blockchain it is possible to correlate the transactions in order to profile users, eventually discovering their identities. It is however possible to fix this issue with the methods discussed in this work, namely mixing services and Bitcoin protocol extensions, which allow to achieve a good level of privacy. Nevertheless, many alternative Blockchain-based and Bitcoin-based cryptocurrencies exist, some of which are specifically designed to assure users privacy, like for instance Monero \cite{getmonero} and ZCash \cite{zcash}. For this reason, if privacy is the main concern, these alternatives could represent a better solution when approaching cryptocurrencies for the first time. 

Eventually, Bitcoin presents a few more issues. The first one is related to the adopted consensus mechanism, namely the proof-of-work, which requires a huge amount of computational effort to be carried out, which translates into a huge amount of resources (electricity in particular) basically being wasted (only one node announces the new mined block). For these reason, cryptocurrencies based on different consensus mechanisms could represent a valid alternative to Bitcoin (like for instance proof-of-stake based currencies). The second issue is instead due to Script, the programming language used to define how bitcoins can be spent. Indeed, as consequence of the limitations of this language, Bitcoin has a very limited support for smart contracts, while other alternative cryptocurrencies (Ethereum \cite{ethereum} in particular) are properly designed to well support them. 
