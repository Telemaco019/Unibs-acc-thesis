\documentclass[12pt, a4paper]{article}
\usepackage[utf8]{inputenc}
\usepackage{listings}
\usepackage{courier}
\usepackage{fancyhdr}
\usepackage{color}
\usepackage{xcolor}
\usepackage{caption}
\usepackage{graphicx}
\usepackage{url}
\usepackage[titletoc]{appendix}
\usepackage{textcomp}
\usepackage{pdfpages}
\usepackage[colorlinks]{hyperref}
\usepackage{amsmath}
\usepackage{amssymb}
\usepackage{booktabs}
\usepackage{graphicx}
\usepackage{xcolor}
\hypersetup{
    colorlinks,
    linkcolor={black},
    urlcolor={blue}
}
\usepackage{tabularx,ragged2e}

\usepackage{amsthm}
\theoremstyle{definition}
\newtheorem{definition}{Definition}[section]

\theoremstyle{remark}
\newtheorem{remark}{Remark}[section]

\usepackage[british,english]{babel}
\usepackage{biblatex}
\addbibresource{bibliography.bib}
\DeclareFieldFormat{url}{Available at: \url{#1}}



\definecolor{light-gray}{gray}{0.95}

\lstset{
 upquote=true,
 showspaces=false,
 showtabs=false,
 frame=none,
 tabsize=2,
 breaklines=true,
 numbers=none,
 showstringspaces=false,
 breakatwhitespace=true,
 escapeinside={(*@}{@*)},
 keywordstyle=\bfseries,
 basicstyle=\footnotesize\ttfamily,
}

\newcommand{\code}[1]{{\footnotesize\texttt{#1}}}

\setlength\parindent{0pt} % Indentazione paragrafi
\setlength{\parskip}{1ex plus 0.5ex minus 0.2ex} % Spaziatura paragrafi

\renewcommand{\labelenumii}{\theenumii}
\renewcommand{\theenumii}{\theenumi.\arabic{enumii}.}

\pagestyle{fancy}
%\renewcommand{\chaptermark}[1]{\markboth{#1}{}}
\renewcommand{\sectionmark}[1]{\markright{\thesection\ #1}}
\fancyhf{}
\fancyhead[LE,RO]{\bfseries}
\fancyhead[LO]{\bfseries\rightmark}
\fancyhead[RE]{\bfseries\leftmark}
\renewcommand{\headrulewidth}{0.5pt}
\renewcommand{\footrulewidth}{0pt}
\addtolength{\headheight}{0.5pt}
\fancypagestyle{plain}{%
	\fancyhead{}
	\renewcommand{\headrulewidth}{0pt}
}
\rfoot{Page \thepage}


\title{
  \textbf{Blockchain and Bitcoin} \\
  \large Overview of Blockchain technology and cryptocurrencies, \\
  focusing on the Bitcoin protocol and its scalability and \\ privacy aspects
}
\date{}
\author{Michele Zanotti}

\begin{document}

  \maketitle
  \tableofcontents

  \begin{enumerate}
    \item Preliminary concepts at the basis of Blockchain \cite{bambara2018blockchain},\cite{bashir2017mastering}
    \begin{enumerate}
      \item Introduction to the cryptography concepts used in Blockchain \cite{bashir2017mastering}
      \begin{itemize}
        \item Cryptography services (confidentiality, authentication, integrity, non-repudiation)
        \item Public and private key cryptography
        \item Elliptic curve cryptography
        \item Hash functions
        \item Elliptic curve digital signature algorithm (ECDSA)
      \end{itemize}
      \item Distributed systems and decentralization
      \item Consensus \& Byzantine generals problem
    \end{enumerate}

    \item Introduction to Blockchain \cite{bambara2018blockchain},\cite{bashir2017mastering}
    \begin{enumerate}
      \item What is a Blockchain
      \item Blockchain features
      \item Types of Blockchain (public, consortium, private)
      \item Blockchain history (why it was invented)
      \item Overview of today Blockchain applications
    \end{enumerate}

    \item Bitcoin \cite{karame2016bitcoin},\cite{antonopoulos2017mastering}
    \begin{enumerate}
      \item Bitcoin protocol specification
      \begin{itemize}
        \item Overview of Bitcoin data types (transaction, scripts, adresses, blocks)
        \item Transactions
        \item Bitcoin network architecture
        \item Bitcoin blockchain (blocks structure, Merkle trees, mining, proof of
        work)
      \end{itemize}
      \item Bitcoin wallets
%      \item Security of Bitcoin transactions
    \end{enumerate}

    \item Bitcoin privacy
    \begin{itemize}
      \item Considerations on user anonymity in Bitcoin
      \item Possible attacks
      \item How to enhance privacy in Bitcoin (explanation of mixing services
      + reference \cite{heilman-blindly-signed-contracts},\cite{saxena-bitcoin-anonymity})
    \end{itemize}

    \item Bitcoin blockchain scalability
    \begin{itemize}
      \item Considerations on the scalability of the Bitcoin blockchain and
      possibile solutions \cite{karame2016bitcoin},\cite{croman-scaling-blockchain}
    \end{itemize}

    \item Alternatives to Bitcoin
    \begin{itemize}
      \item Bitcoin limitations
      \item Alternatives to proof of work \cite{bentov-no-proof-of-work}
      \item Namecoin
      \item Litecoin
      \item ZCash
    \end{itemize}

%    {
%    \color{gray}

%    \item Smart contracts
%    \begin{itemize}
%      \item History
%      \item What smart contracts are
%      \item Security
%    \end{itemize}

%    \item Ethereum
%    \begin{itemize}
%      \item History
%      \item Ethereum components (Keys, addresses, accounts)
%      \item Ethereum blockchain
%      \item Ethereum network
%      \item Ethereum transactions
%      \item Ethereum virtual machine
%    \end{itemize}


%    \item (Possibly) Pratical example of blockchain application:
%    development of a distributed application through Ethereum Solidity
%    (Something similar to the application development shown in reference \cite{bambara2018blockchain})
%    }
  \end{enumerate}

  \section{Introductory concepts}

%%%%%%%%%%%%%%%%%%%%%%%%%%%
% *** FIRST SUB-SECTION ***
%%%%%%%%%%%%%%%%%%%%%%%%%%%
\subsection{Hash functions}
A hash function is a fuction that maps an arbitrary long input string to a fixed
length output string. Let $h$ refer to an hash function of length $n$:

\[
  h\colon \{0,1\}^* \to \{0,1\}^n
\]

$m$ is usually called ``the message'', while $d$ is usually called ``the digest``
and it can be seen as a compact representation of m. The length of $d$ is the

Hash functions are usually used to provide data integrity and they're also used to
length of the hash.
construct other cryptographic primitives such as MACs and digital signatures.

\subsubsection{Desired properties}
An hash function should ideally meet these properties:
\begin{itemize}
  \item \textbf{Computational efficiency}: given m, it must be easy to compute ${d=h(m)}$
  \item \textbf{Preimage resistance} (also called \textbf{one-way property}):
  given ${d=h(m)}$, it must be computationally infeasible computing $m$ ($m$ is the
  preimage)
  \item \textbf{Weak collision resistance} (also called
  \textbf{2\textsuperscript{nd} preimage resistance}): given $m_1$ and ${d_1=h(m_1)}$,
  it must be computationally infeasible finding a $m_2 \neq m_1$ so that ${h(m_2)=d_1}$
  \item \textbf{Strong collision resistance}: it must be computationally infeasible
  finding paris of distinct and colliding messages. Two messages $m_1\neq m_2$
  collide when ${h(m_1)=h(m_2)}$.
  \item \textbf{Avalanche effect}: changing a single bit of $m$ should cause every
  bit of ${d=h(m)}$ to change with probability ${P=0.5}$
\end{itemize}

\subsubsection{Examples of hash functions}
\begin{itemize}
  \item \textbf{MD5}: published in 1991, it's a 128-bit hash function that was
  used for file integrity checks. Today it's considered unsecure and it shouldn't
  be used anymore.
  \item \textbf{Secure Hash algorithm 1 (SHA-1)}: 160-bit hash function that was
  used in SSL and TLS implementations. Today is considered unsecure and it's
  deprecated.
  \item \textbf{SHA-2}: family of SHA functions which includes SHA-256, SHA-384
  and SHA-512. SHA-256 is currently used in several parts of the Bitcoin network.
  \item \textbf{SHA-3}: latest family of SHA functions, it is a NIST-standardized
  version of Keccak, which uses a new approach called ``sponge construction''
  instead of the Merkle-Damgard transformation previously used. This family
  includes SHA3-256, SHA3-384 and SHA3-512.
\end{itemize}

\subsubsection{Design of SHA-256}

\subsubsection{Message Authentication Codes (MACs)}
A MAC is an hash function which uses a key and which can therefore be used to
provide both integrity and authentication (proof of origin). Authentication is
based on a key pre-shared between the sender and the receiver. The receiver can
verify both integrity and authentication of a message by computing the MAC function
of the message and comparing it with the one received from the sender: if they are the
same then integrity and authentication are confirmed (note that it is assumed that
only the sender and the receiver know the key).

MAC functions can be constructed using block ciphers or hash functions:
\begin{itemize}
  \item in the first approach, block ciphers are used in the Cipher block chaining mode (CBC mode):
  the MAC of a message will be the output of the last round of the CBC operation.
  The length of MAC in this case is the same as the block length of the block cipher
  used to generate it.
  \item In the second approach they key is hashed with the message using a certain
  construction scheme. The most simple ones are \emph{suffix-only} and
  \emph{prefix-only}, which however are weak and vulnerable:
  \begin{itemize}
    \item suffix-only: ${d=MAC_k(m)=h(m|k)}$, where $h$ is an hash function
    \item prefix-only: ${d=MAC_k(m)=h(k|m)}$, where $h$ is an hash function
  \end{itemize}
\end{itemize}





\subsection{Digital signature}
Digital signatures are used to associate a message with the entity from which the
message has been originated. They provide the same service as MACs (authentication
and non-repudiation) plus the non-repudiation.

Digital signature is based on public key cryptography: Alice can sign a message
by encrypting it using its private key. Usually however, for efficiency and security
reasons, Alice doesn't encrypt the message but its digest (hash of the message).
Figure \ref{fig:digital-signature} shows how a generical digital signature function
works.

An example of digital signature algorithms are RSA and ECDSA.

\begin{figure}[!htb]
	\centering
	\includegraphics[width=1\linewidth]{img/digital-signature.png}
	\caption{digital signature signing and verification scheme}
	\label{fig:digital-signature}
\end{figure}





\subsection{Elliptic Curve Digital Signature Algorithm (ECDSA)}
ECDSA is a variant of the Digital Signature Algorithm (DSA) which uses elliptic
curve cryptography.

\subsubsection{Key pair generation}
\begin{enumerate}
  \item Define an elliptic curve $E$ with modulus $P$, coefficients $a$ and $b$ and a
  generator point $A$ that forms a cyclic group of order $p$, with $p$ prime
  \item Choose a random integer $d$ so that ${0 < d < q}$
  \item Compute the public key $B$ so that ${B = d  A}$
\end{enumerate}
The public key is the sextuple ${K_{pb} = (p,a,b,q,A,B)}$, while the private key
is the value of $d$ randomly chosen in Step 2: ${K_{pr} = d}$

\subsubsection{Signing a message}
\begin{enumerate}
  \item Choose an ephemeral key $K_e$, where ${0 < K_e < q}$.
  It should be ensured that $K_e$ is truly random and no two signatures have the same key
  because otherwise the private key can be calculated
  \item Compute ${R = K_e A}$
  \item Initialize a variable $r$ with the x coordinate value of the point $R$
  \item The signature on the message $m$ can be calculated as follow:
  \[{S=(h(m)+d r)K_e^{-1}\bmod q}\]
  where $h(m)$ is the hash of the message $m$. The signature is the pair ${(S,r)}$.
\end{enumerate}

\subsubsection{Signature verification}
A signature can be verified as follow:
\begin{enumerate}
  \item Compute ${w=S^{-1}\bmod q}$
  \item Compute ${u_1=w  h(m)\bmod q}$
  \item Compute ${u_2=w  r \bmod q}$
  \item Calculate the point ${P=u_1  A + u_2  B}$
  \item The signature ${(S,r)}$ is accepted as a valid signature only if:
   \[{X_P=r \bmod q}\]
   where $X_P$ is the x-coordinate of the point P calculated in Step 4
\end{enumerate}











%%%%%%%%%%%%%%%%%%%%%%%%%%%
% *** SECOND SUB-SECTION ***
%%%%%%%%%%%%%%%%%%%%%%%%%%%
\subsection{Distributed systems}
\subsubsection{What is a distributed system}
Blockchain at its core is basically a distributed system, therefore it is essential
to understand distributed systems before understanding Blockchain.

A distributed system is a network that consists of autonomous nodes, connected
using a distribution middleware, which act in a coordinated way (passing message
to each other) in order to achieve a common outcome and that can be seen by the
user as a single logical platform.

A node is basically a computer that can be seen as an individual player inside
the distributed system and it can be honest, faulty or malicious. Nodes that
have an arbitrary behavior (which can be malicious) are called \emph{Byzantine nodes}.

\begin{figure}[!htb]
	\centering
	\includegraphics[width=0.6\linewidth]{img/distributed-system.png}
	\caption{design of a distributed system. N4 is a Byzantine node while L1 is a
  broken/slow network link}
	\label{fig:distributed-system}
\end{figure}

The main challenge in a distributed system is the fault tollerance: even if some
of the nodes fault or links break, the system should tollerate this and should
continue to work correctly. There are essentially two types of fault: a simple node
crash or the exhibition of malicious or inconsistent behavior arbitrarily. The
second case is the most difficult to deal with and it's called \emph{Byzantine
fault}. In order to achieve fault tolerance, replication is usually used.

Desired properties of a distributed system are the following:
\begin{itemize}
  \item \textbf{Consistency}: all the nodes have the same lates available copy of
  the data. It is usually achieved through consensus algorithms which ensure that
  all nodes have the same copy of the data
  \item \textbf{Availability}: the system is always working and responding to the
  input requests without any failures
  \item \textbf{Partition tolerance}: if a group of nodes fails the distributed
  system still continues to operate correctly
\end{itemize}
There is however a theorem, the \emph{CAP theorem}, which states (and proves)
that a distributed system cannot have all these three properties at the same time.
In particular, the theorem states that in the presence of a network partition (due
for example to a link failure) one has to choose between consistency and availability.


\subsubsection{The Byzantine Generals Problem (BGP)}
The Byzantine Generals Problem (BGP) is a problem described by Leslie Lamport
\cite{lamport1982byzantine} in which a group of generals, each one leading a portion
on the Byzantine army, are surrounding a city and they have to formulate a plan
for attacking it (simplifying, they have to decide wheter to attack or retreat
from the city). Their only communication way is the messenger and they have to
agree on a common decision. The issue is that some of the generals may be
traitors trying to prevent the loyal generals from reaching an agreement by
communicating a misleading message. The generals need an algorithm to guarantee
that all the generals agree on the same plan (attack or retreat) regardless of what
traitors generals do. Loyal generals will always do what the algorithm says they
should, while the traitors may do anything they wish.

As an analogy with distributed systems:
\begin{itemize}
  \item generals can be considered as nodes
  \item traitors can be considered Byzantine nodes
  \item the messenger can be seen as the channels of communication between the generals.
\end{itemize}

\subsubsection{Consensus}
Consensus is the process of agreement between untrusted nodes on a data value.
When the involved nodes are only two it's really easy to achieve consensus, while
in a distributed system with more than two nodes it is really hard (in this case
the process of achieving consensus is called \emph{distributed consensus}).

A consensus mechanism must meet these requirements:
\begin{itemize}
  \item \textbf{Agreement}: all the nodes must agree on the same value
  \item \textbf{Termination}: the execution of the consensus process must come
  to an end and the nodes have to reach a decision
  \item \textbf{Validity}: the agreed value must have been proposed by at least
  one honest node
  \item \textbf{Fault tolerance}: the consensus algorithm must be able to run even
  in the presence of one or more Byzantine (faulty or malicious) nodes
  \item \textbf{Integrity}: the nodes make decisions only once in a single
  consensus cycle (in a single cycle a node cannot make the decision more than once).
\end{itemize}

  \section{First}
asdasd
\subsection{test}

  \section{Bitcoin}\label{sec:Bitcoin}
%%%%%%%%%%%%%%%%%%%%%%%%%%%
% *** FIRST SUB-SECTION ***
%%%%%%%%%%%%%%%%%%%%%%%%%%%
\subsection{Introduction}

Bitcoin it's the first fully decentralized cryptocurrency. It was invented by
Satoshi Nakamoto in 2008 and it was the first real implementation of Blockchain.
Bitcoin can be either defined as a protocol, a digital currency and a platform.

Bitcoin can be seen as a combination of
\vspace{-\topsep}
\begin{itemize}
  \item[-] a decentralized peer-to-peer-network (the Bitcoin protocol)
  \item[-] a public transaction ledger (the blockchain)
  \item[-] a set of rules for validating transactions (consensus rules)
  \item[-] a mechanism for reaching distributed consensus on the blockchain (distributed
  consensus algorithm)
\end{itemize}
\vspace{-\topsep}
that allows the usage of the digital currency named bitcoin.

From now on, Bitcoin with the capital $B$ will refer to the Bitcoin protocol
while bitcoin with the lowercase $b$ will refer to the bitcoin currency.



Bitcoin is a distributed peer-to-peer system in which users can exchange
currency over the network just as it can be done with conventional currency.
However, unlike traditional currencies, bitcoins are enterely virtual and thus
there are no physical coins. In particular, there are not even virtual coins since
they are implied in the transactions that send value from a sender to a receiver:
users have private keys which allow them to prove the ownership of bitcoins and
sign transactions in order to unlock the value and transfer it to another user.
These keys are the only requirement for spending bitcoins and therefore they are
protected in wallets stored in the user's devices.

\subsubsection*{The reference implementation}
Bitcoin is an open source project and is developed by a community of volunteers.
The first implementation was released by Satoshi Nakamato in 2008 (the only member
of the development community at the time). That implementation during the years
has been heavily modified and improved evolving into what is known as \emph{Bitcoin
Core}, which is now the reference implementation of the Bitcoin system. This
implementation is considered the authoritative one and it specifies how each
part of the system has to be implemented.










%%%%%%%%%%%%%%%%%%%%%%%%%%%
% *** THIRD SUB-SECTION ***
%%%%%%%%%%%%%%%%%%%%%%%%%%%
\subsection{Scripts} \label{sec:scripts}Bitcoin uses a simple stack-based
programming language called ``Script'' for describing how bitcoins can be spent
and transferred in order to extend flexibility and support different types of
transactions. Essentially, a Bitcoin script a list of instructions recorded with
each transaction that describe how the next person wanting to spend the Bitcoins
being transferred can gain access to them \cite{script-bitcoin-wiki}.

Script is a very simple language and it's not Turing complete. The language has
been deliberately designed limiting its operators (it doesn't have loop operators
and complex control flow different than conditional control flow) in order to
avoid abuses of the scripts for conducting denial of service attacks, since the
transaction scripts has to be executed on each node of the network.

Script supports a number of function called ``Opcodes'', uses a reverse polish
notation in which every operand is followed by its operators and it's evaluated
from the left to the right using a LIFO stack. Table \ref{tab:opcode-example} shows the most common
Opcodes while figure \ref{fig:script-example} shows an example of Script program.



\begin{table}[!ht]
\footnotesize
\begin{tabularx}{\textwidth}{l X}
\hline
\textbf{Opcode} & \textbf{Description}  \\\hline
OP\_CHECKSIG & This takes a public key and signature and validates the signature of the hash of the transaction. If it matches, then TRUE is pushed onto the stack; otherwise, FALSE is pushed.  \\
\\
OP\_EQUAL & This returns 1 if the inputs are exactly equal; otherwise, 0 is returned.  \\
\\
OP\_DUP & This duplicates the top item in the stack.  \\
\\
OP\_HASH160 & The input is hashed twice, first with SHA-256 and then with RIPEMD-160. \\
\\
OP\_VERIFY & This marks the transaction as invalid if the top stack value is not true. \\
\\
OP\_EQUALVERIFY & This is the same as OP\_EQUAL, but it runs OP\_VERIFY afterwards. \\
\\
OP\_CHECKMULTISIG & This takes the first signature and compares it against each public key until a match is found and repeats this process until all signatures are checked. If all signatures turn out to be valid, then a value of 1 is returned as a result; otherwise, 0 is returned. \\
\hline
\end{tabularx}
\caption{\footnotesize Most commonly used Opcodes. Taken from the
bitcoin developer's guide \cite{bitcoin-developer-guide}. }
\label{tab:opcode-example}
\end{table}








%%%%%%%%%%%%%%%%%%%%%%%%%%%
% *** THIRD SUB-SECTION ***
%%%%%%%%%%%%%%%%%%%%%%%%%%%
\subsection{Keys and Addresses}
As mentioned in this chapter's introduction, ownership of bitcoin is established
through digital keys, bitcoin addresses, and digital signatures.

In order to be included in the Bitcoin blockchain, transactions require a valid
signature which can be generated only with a private (secret) key. The private
key therefore proves the ownership of bitcoins by signing transactions and
transferrig value from a user to another. Keys come in pairs consisting of a
private (secret) key and a public key and they are generated through Elliptic
Curve Cryptography. In analogy with the traditional banking,the public key can be
seen as the bank account number while the private key as the secret PIN (or the
signature on a check) which provides control over the account by allowing to unlock
the value and transferring it to other people.

\subsubsection{Adresses} An address is unique string of digits and characters
which identify the originator and/or the destination of a transaction. Addresses
are derived from public keys through one-way cryptographic hashing in order to
obtain the public key fingerprint. In particular, a Bitcoin address is derived
by hashing the user's public key it twice, first with the SHA-256 algorithm and
then with RIPEMD160. This produces a 160-bit hash, which is then prefixed with a
version number and finally encoded using Base58Check encoding. The final result
is a 26-35 chatacters string which begins with ``$1$'' (public key address) or
``$3$'' (pay-to-script-hash address) and it looks like the the string below:
\begin{center} \code{1J7mdg5rbQyUHENYdx39WVWK7fsLpEoXZy} \end{center} The
generation process scheme is shown in figure \ref{fig:address-generation}.

\begin{figure}[!htb]
	\centering
	\includegraphics[width=0.85\linewidth]{img/address-generation.png}
	\caption{Bitcoin address generation scheme}
	\label{fig:address-generation}
\end{figure}

\paragraph{Base58 and Base58Check} Base58 is an encoding scheme which allows to
represent long numbers as alphanumeric strings. It is a subset of Base64, which
represent numbers using 26 lowercase letters, 26 capital letters, 10 numerals,
and 2 more ``special'' characters and it's usually used to encode email
attachments. In particular, Base58 is Base64 without all that characters that
are frequently mistaken for one another, namely it is Base64 without the 0
(number zero), O (capital o), l (lower L), I (capital i) and the two special
characters. Base58Check is a Base58 encoding with an additional checksum of four
bytes added to the end of the data that is being encoded which prevents a
mistyped bitcoin address from being accepted by the wallet software as a valid
destination.


\paragraph{P2SH and P2PKH}  As already mentioned before, Bitcoin addresses that
begin with the number ``$3$'' are pay-to-script hash (P2SH) addresses. Unlike
the address which start with ``$1$'', also known as pay-to-public-key-hash
(P2PKH), which are associated to a public key owned by a user, the P2SH
addresses designate the beneficiary of a Bitcoin transaction as the hash of a
script. When a user send a bitcoin to a P2PKH address, that bitcoin can only
be spent by the receiver by presenting the corresponding private key signature
and public key hash associated to its address. When instead the bitcoin is sent to
a P2SH address, namely to the hash of a script, the requirements for spending that
bitcoin are defined by the script and are usually more restrictive (for example it
could be required more than one signature to prove the ownership). A P2SH address
is derived from a transaction script in the same way a P2PKH address is derived
from a public key (double hashing + Base58Check encoding).


\subsubsection{Keys}
Public and private keys in Bitcoin are generated through ECC and they can be
represented in different formats. All the possible representations, even if they
look different, correspond to the same number. This has been done in order to
facilitate people to read and transcribe the keys without introducing errors.

\paragraph{Private keys}
Private keys are simply a 256-bit random number. For generating it, Bitcoin
software uses the underlying operating system’s random number generators which
usually is initialized by a human source of randomness, like for example the elapsed
time between the pression of the keys of the keyboard.

\paragraph{Private key formats}
The private key can be represented in different formats (shown in table
\ref{tab:private-key-formats}), each one corresponding to the same 256-bit number.
Different formats are used in different circumstances: for example Hexadecimal
and raw binary formats are used internally in software while WIF is used by users.
\begin{table}[h!]
\centering
\resizebox{\textwidth}{!}{%
\begin{tabular}{@{}lll@{}}
\toprule
\multicolumn{1}{c}{\textbf{Type}} & \multicolumn{1}{c}{\textbf{Prefix}} & \multicolumn{1}{c}{\textbf{Description}}                                     \\ \midrule
Raw                               & None                                & 32 bytes                                                                     \\
Hex                               & None                                & 64 hexadecimal digits                                                        \\
WIF                               & 5                                   & Base58Check encoding \\
WIF-compressed                    & K or L                              & As above, with added suffix 0x01 before encoding                             \\ \bottomrule
\end{tabular}%
}
\caption{Private key representation formats \cite{antonopoulos2017mastering}}
\label{tab:private-key-formats}
\end{table}

\paragraph{Public key generation}
Public keys are generated starting from the private keys using elliptic curve
multiplication, which is a so-called ``trap door'' function: it is easy to do in
one direction (multiplication) and impossible to do in the reverse direction (division).
Bitcoin uses the elliptic curve and the set of constants specified by the secp256k1
standard, defined by the NIST. The elliptic curve used is defined by the following
equation:
\begin{equation}\label{eq:bitcoin-curve}
  y^2 = (x^3 + 7)~\text{over}~(\mathbb{F}_p)
\end{equation}
\begin{center}
  or, equivalently:
\end{center}
\begin{equation}
  y^2 \bmod p = (x^3 + 7) \bmod p
\end{equation}
where $p = 2^{256} – 2^{32} – 2^9 – 2^8 – 2^7 – 2^6 – 2^4 – 1$ is a very large prime number.
Starting from the private key $k$, the public key $K$ is calculated multiplying
it by a predetermined point on the curve called the generator point $G$ (defined
by the secp256k1 standard) in order to produce another point somewhere else on
the curve, which will correspond to the public key $K$:
\[K = k * G\]
Since the generator point $G$ is always the same for all bitcoin users, a
private key $k$ multiplied with $G$ will always result in the same public key $K$.
The relationship between $k$ and $K$ is fixed and known but it can only be
calculated in one direction (from $k$ to $K$), so it's impossible to derive from
an address (derived from K) the corresponding user's private key.

\paragraph{Public key formats} In Bitcoin, since ECC is used, a public key in
the uncompressed format is a point on an elliptic curve consisting of the
coordinates pair $(x,y)$. Uncompressed public keys are presented with the prefix
\code{04} followed by two 256-bit numbers, one for each coordinate, and
therefore they are 65 Bytes long. The compress format instead includes only the
x-coordiante since the y one can be derived from it and by solving the equation
\eqref{eq:bitcoin-curve} it uses the prefixes \code{03}, if the y-coordinate is an
odd number, or \code{02}, if it is an even number. The length of a compressed
public key is therefore 33 Bytes. Compressed public keys were introduced in
order to reduce the size of the transactions, since the most of them also include
the public key. The reason why two different prefixes are required for
compressed keys is that the left side of the equation \eqref{eq:bitcoin-curve} is
$y^2$ and therefore the solution for $y$ is a square root, which can have a
``positive'' or ``negative value'': graphically, this means that the
y-coordiante can either be above or below the x-axis and therefore two different
points can be identied since the curve is symmetric. Actually since we are in
fhe field $\mathbb{F}_p$ it doesn't make sense talking about positive and
negative values: the y-coordinate can in fact be \emph{even} or \emph{odd}
(which correspond to the positive/negative terms used before).

Note that a a public key in both compressed and uncompressed formats always
corresponds to the same private key, even if the two formats have a different
representation. The address derived from the compressed public key however is
different from the address derived from the uncompressed one. To solve this
issue, compressed private keys have been introduced: a compressed private key is
a ``private key from which only compressed public keys should be derived'',
while uncompressed private keys are ``private keys from which only uncompressed
public keys should be derived'' \cite{antonopoulos2017mastering}.











%%%%%%%%%%%%%%%%%%%%%%%%%%%
% *** FOURTH SUB-SECTION ***
%%%%%%%%%%%%%%%%%%%%%%%%%%%
\subsection{Transactions}\label{sec:transactions} Transactions are data structures that encode the
transfer of value between participants in the bitcoin system. In important to
point out that they are not encrypted and are publicly visible in the
blockchain. Blockchain blocks are made up of transactions and these can be
viewed using any online blockchain explorer.

\subsubsection{Transaction inputs and outputs}
A transaction includes at least one input and output: inputs can be seen as
coins being spent that the user has created in a previous transaction while
outputs as coins being created.

\paragraph{Outputs and UXTO} In particular, outputs are discrete and indivisible
units of bitcoin measured in \emph{Satoshi}\footnote{One Satoshi
$=10^{-8}$bitcoins}, recorded on the blockchain and recognized as valid by the
network. All the available and spendable outpus are stored in the blockchain and
they are called \emph{unspent transaction outputs} or \emph{UTXO}. The balance
shown by Wallets application is nothing more than the aggregated value all the
UTXOs the user can spend with the keys it controls. Note that a UXTO can only be
spent in its entirety by a transaction, consequently, if an UTXO is larger than
the desired value of a transaction, it must still be consumed in its entirety
and change must be generated in the transaction (most of the bitcoin
transactions generate change). Transaction outputs consist of two parts: an
amount of bitcoin (expressed in Satoshis) and a cryptographic puzzle that
determines the conditions required to spend the output. This puzzle is also
known as a \emph{locking script} and it consists of a digital signature and
public key proving the ownership of the UXTO.

\paragraph{Inputs} Transaction inputs consists of which UTXO will be consumed
(can be more than one single UXTO) and a proof of ownership through the
unlocking script to unlock the selected UXTO.


\subsubsection{Transactions structure}

\begin{table}[h!]
\centering
\resizebox{\textwidth}{!}{%
\begin{tabularx}{\textwidth}{l l X}
\toprule
\textbf{Field} & \textbf{Size} & \textbf{Description}                                    \\ \midrule
Version Number                    & 4 bytes                             & Used to specify rules to be used by the miners and nodes for transaction processing.                                                                     \\
\\
Input counter                     & 1-9 bytes                           & The number of inputs included in the transaction.                                                       \\
\\
List of inputs                    & variable                            & Each input is composed of several fields, including Previous transaction hash, Previous Txout-index, Txin-script length, Txin-script, and optional sequence number. The first transaction in a block is also called a coinbase transaction. It specifies one or more transaction inputs. \\
\\
Output counter                    & 1-9 bytes                           & A positive integer representing the number of outputs.                        \\
\\
List of Outputs                   & variable                            & Outputs included in the transaction. \\
\\
lock\_time                         & 4 bytes                             & This defines the earliest time when a transaction becomes valid. It is either a Unix timestamp or a block number.                             \\ \bottomrule
\end{tabularx}
}
\caption{Structure of Bitcoin transactions}
\label{tab:transaction-structure}
\end{table}


\subsubsection{Transactions life cycle}
This is the typical life cycle of a transaction:
\begin{enumerate}
  \item A sender sends a transaction (using a wallet application)
  \item The wallet signs the transaction using the sender's private key in order
  to proof the ownership of the value being transferred
  \item The transaction is broadcasted to the Bitcoin network using a flooding algorithm.
  \item Mining nodes include this transaction in the next block to be mined.
  \item Once a mining node solves the Proof of Work problem it broadcasts the
  newly mined block to the network and the confirmation process starts: each
  nodes verify the block and propagate it further
  \item The receiver start to receive confirmations. After approximately six
  confirmations, the transaction is considered finalized and confirmed.
\end{enumerate}




\subsubsection{Transaction fees} Most transactions include transaction fees.
These fees have to purposes: compensate the bitcoin miners and act as a security
mechanism by making economically infeasible for attackers to flood the network
with transactions. The value of the fees dependens on the size of the
transaction since it's calculated by subtracting the sum of the outputs to the
sum of the inputs: \[Fees = Sum(Inputs) – Sum(Outputs)\] Fees also  act as an
incentive for miners to encourage them to include a user transaction in the
block the miners are creating. Each miner chooses from a memory pool which
transactions include in the block he will propose based on their priority:  a
transaction with a higher fee will be picked up sooner by the miners since it's
more profitable.


\subsubsection{Coinbase transactions} A particual kind of transaction is the
\emph{coinbase transaction}, which is created by the ``winning'' a miner and is
the first transaction in a block. This transactions create brand-new bitcoins
that the miner can spend as a reward for mining and do not consume UTXO,
instead, they have a special type of input called the \emph{coinbase}.














%%%%%%%%%%%%%%%%%%%%%%%%%%%
% *** FIFTH SUB-SECTION ***
%%%%%%%%%%%%%%%%%%%%%%%%%%%
\subsection{The Bitcoin Blockchain} The Bitcoin blockchain is a linked list of
blocks of transactions, each one  identified by a SHA-256 hash. Each block
references the previous one (the \emph{parent block}) by embedding its hash in
the header. This chains of hashas goes back all the way to the first block ever
created, known as the \emph{genesis block}.

Although a block can have only one single parent, it can temporarily have
multiple childrens. This happen during a \emph{blockchain fork}, a temporary
situation which occurs when miners solve the proof of work of their block almost
simultaneously. Eventually however the forks are resolved and  only one child
block becomes part of the blockchain.

Modifying a block causes it hash to change. Consequently, since each block
contains in its header the hash of its parent block, changing a block causes
the child’s hash to change, which also requires a change in its child block hash
and so on. This cascade effect ensures that once a block has many generations
following it, it cannot be changed without forcing a recalculation of all
subsequent blocks: since this recalculation requires a huge computation, the
blockchain history is pratically immutable. This is a key feature of the Bitcoin
security.

\begin{figure}[!htb]
	\centering
	\includegraphics[width=0.9\linewidth]{img/bitcoin-blockchain-scheme.png}
	\caption{Bitcoin blockchain structure scheme}
	\label{fig:bitcoin-blockchain}
\end{figure}


\subsubsection{The block structure} Table \ref{tab:bloock-structure} summarize
the structure of a block of the blockchain, while table \ref{} shows the structure
of the block header.

\begin{table}[h!]
\centering
\resizebox{\textwidth}{!}{%
\begin{tabularx}{\textwidth}{l l X}
\toprule
\textbf{Field} & \textbf{Size} & \textbf{Description}     \\ \midrule
Block size  & 4 bytes  & The size of the block, in bytes.       \\
\\
Block header & 80 bytes & Several fields form the block header. \\
\\
Transactions counter & 1-9 bytes & How many transactions the block contains. \\
\\
Transactions & Variable & The transactions recorded in the block \\ \bottomrule
\end{tabularx}
}
\caption{Structure of a Bitcoin block}
\label{tab:bloock-structure}
\end{table}



\begin{table}[h!]
\centering
\resizebox{\textwidth}{!}{%
\begin{tabularx}{\textwidth}{l l X}
\toprule
\textbf{Field} & \textbf{Size} & \textbf{Description}     \\ \midrule
Version  & 4 bytes  & A version number to track software/protocol upgrades.       \\
\\
Previous block hash  & 32 bytes & A reference to the hash of the previous (parent) block in the chain. \\
\\
Merkle root & 32 bytes & A hash of the root of the merkle tree of this block’s transactions. \\
\\
Timestamp & 4 bytes & The approximate creation time of this block (seconds from Unix Epoch). \\
\\
Difficulty target & 4 bytes & The Proof-of-Work algorithm difficulty target for this block. \\
\\
Nonce & 4 bytes & A counter used for the Proof-of-Work algorithm. \\ \bottomrule
\end{tabularx}
}
\caption{Structure of a Bitcoin block header}
\label{tab:bloock-header-structure}
\end{table}


\subsubsection{Merkle trees} Each block summmarize all the transactions it
contains using a Merkle tree, which is a data structure used for efficiently
summarizing and verifying the integrity of large sets of data. A Merkle tree  is
a binary tree containing hashes and it produces an overall digital fingerprint
of the entire set of transactions, providing a very efficient method to verify
whether a transaction is included in a block. The hash algorithm used in
bitcoin’s merkle trees is double-SHA256 (SHA256 applied twice).

In the Bitcoin blocks headers only the 32-byte hash corresponding to the tree
root is stored, which summarizes all the transactions and allows a node to check
whether a specific transaction is included in the block by computing the
$log_2(N)$ hashes which make up a \emph{merkle path} connecting the transaction
to the root of the tree, with $N$ number of transactions of the block. Figure
\ref{fig:merkle-tree-path} shows an example of merkle path, while table
\ref{tab:merkle-tree-sizes} compares the size of a block to the size of a merkle
path.

 Thanks to merkle trees, a node can download just the block headers (80 bytes
 per block) and still be able verify whether a transaction is included in a
 block by retrieving a small merkle path from a full node (which stores the
 complete blockchain) instead of storing or retrieving the full block, which
 is a lot more efficient as pointed out by table \ref{tab:merkle-tree-sizes}.

 The nodes that do not maintain a full copy of the blockchain are \emph{called
 simplified payment verification} (SPV) nodes and they use merkle paths to
 verify transactions without downloading full blocks.

 \begin{table}[h!]
 \footnotesize

 \centering
 \resizebox{\textwidth}{!}{%
 \begin{tabularx}{\textwidth}{l l l l}
 \toprule
 \textbf{Number of transactions}	& \textbf{Approx size of block} & \textbf{Path size} &	\textbf{Path size}   \\ \midrule
 16 transactions & 4 kilobytes & 4 hashes & 128 bytes \\
 \\
 512 transactions & 128 kilobytes & 9 hashes & 288 bytes \\
 \\
 2048 transactions & 512 kilobytes & 11 hashes & 352 bytes \\
 \\
 65535 transactions & 16 megabytes & 16 hashes & 512 bytes \\
 \bottomrule
 \end{tabularx}
 }
 \caption{Merkle tree efficiency}
 \label{tab:merkle-tree-sizes}
 \end{table}


\begin{figure}[!htb]
	\centering
	\includegraphics[width=1\linewidth]{img/merkle-tree-path.png}
	\caption{Example of a Merkle path. The path consists of the four hashes with the
  blue background and with these hashes any node can prove that $H_K$ is included
  in the merkle root by computing four additional pair-wise hashes outlined in a
  dashed line.  }
	\label{fig:merkle-tree-path}
\end{figure}








%%%%%%%%%%%%%%%%%%%%%%%%%%%
% *** SIXTH SUB-SECTION ***
%%%%%%%%%%%%%%%%%%%%%%%%%%%
\subsection{The Bitcoin network}

\subsubsection{Node types}
\paragraph{Full node} A full node is a node that maintains the full copy of the
blockchain, validates all incoming transactions and blocks and forwards
transactions and blocks to its peers.

\paragraph{Miner} Miners are nodes that compete to each other to solve the
proof-of-work and create new blocks. Some miner are also full nodes which
mantain a full copy of the blockchain, while others don't and instead they
partecipate to a mining pool which depends on a pool server which maintains a
full copy of the blockchain.

\paragraph{Lightweight clients} Lightweight clients are clients that do not
store, nor maintain the full Bitcoin blockchain, but follow a simple payment
verification (SPV) scheme that allows them to verify that a transaction has been
included in the blockchain by receiving and verifying only the block headers
relevant to their wallets. They don't need to perform transaction or block
validation.

\subsubsection{Network architecture}
The Bitcoin network architecture is a Peer-to-Peer (P2P) network on top of the
Internet. P2P means that all the nodes of the network are peers to each other:
they are all equal, there aren't ``special'' nodes, servers, centralized services
and heiarchies. The nodes are interconnected in a mesh network with a ``flat''
topology and the nodes both provide and consume services at the same time.  









%%%%%%%%%%%%%%%%%%%%%%%%%%%
% *** SEVENTH SUB-SECTION ***
%%%%%%%%%%%%%%%%%%%%%%%%%%%
\subsection{Mining and Proof of Work} Mining is a resource-intensive
process by which transactions are validated and new blocks are added to the
blockchain. Transactions that become part of a block and added to the blockchain
are considered confirmed, which means that the receivers of the transactions can
spend the value they received.

Roughly one new block is created (\emph{mined}) every 10 minute and Miners after
mining a block are rewarded with two types of rewards: new coins created with
each new block (a basecoin transaction) and transaction fees from all the
transactions included in the block.

\paragraph{Proof of Work} In order to earn the reward, miners compete with each
other to solve an hard problem based on a cryptographic hash algorithm. The
solution to the problem, called the Proof-of-Work, is included in the new mined
block and acts as proof that the miner expended significant computing effort.
The proof of work requirement is given by the following equation:
\begin{equation}
  H ( N || Prev\_hash || Tx || Tx || . . . Tx) < Target
\end{equation} where H is the SHA256 hash function, N is the nonce
contained in the block header, $Prev\_hash$ is the hash of the previous block,
Tx are the transactions cointained in the block, $Target$ is the difficulty
value and $||$ is the concatenate operator.
For example, if the target is $0x10000000000000$ then finding a hash less than
the target means finding a hash that stats with a zero. Consequently, the
difficulty level of the proof of work can be seen as the number of zeros that the
hash of the block has to start with. The only way for finding a valid hash therefore
is to use a the brute force method, changing the nonce value for every hash
calculation in order to get different hashes until a valid one in found
(any specific hash input to one and only one hash value).
Once the miner met the correct number of zeros, the block is immediately
broadcasted and accepted by other miners.
The difficulty of this work is always adjusted (increased) so as to limit the
rate at which new blocks can be generated by the network to one every 10 minutes.

The algorithm for mining a block can be summarized in the following steps:
\begin{enumerate}
  \item Retrieve the hash of the previous block from the Bitcoin network
  \item Choose wich transaction include in the block (according to their priority)
  \item Compute the double SHA256 hash of the block header
  \item Check whether the resultant hash is lower than the current difficulty level
  (target). If so, then stop the process, otherwise change the nonce (usually it is
  increased by 1) and go back to step 3.
\end{enumerate}















%%%%%%%%%%%%%%%%%%%%%%%%%%%
% *** EIGHTH SUB-SECTION ***
%%%%%%%%%%%%%%%%%%%%%%%%%%%
\subsection{Consensus} Mining is a key feature of Bitcoin which secures the
bitcoin system and allows to have network-wide consensus without a central
authority. In particular, in Bitcoin consensus is not achieved explicitly since
there is no election or fixed moment when consensus occurs. Instead, consensus
is an emergent artifact of the asynchronous interaction of thousands of
independent nodes. For this reason, in Bitcoin the consensus process is called
\emph{emergent consensus}. Bitcoin’s decentralized consensus emerges from
four processes that occur independently on nodes across the network:
\begin{itemize}
  \item Independent verification of each transaction by every full node
  \item Independent aggregation of verified transactions into new blocks by mining
  nodes and inclusion of the proof of work
  \item Independent verification of the new blocks by every node and assembly
  into the chain: each node performs a series of tests for validating it before
  propagating it to its peers and inserting it into the blockchain.
  This ensures that only valid blocks are propagated on the network: block which
  are tampered with will thus be rejected.
  Thanks to this verification, dishonestly miner (for example miners who write
  themselves a transaction for an arbitrary amount of bitcoin instead of the correct rewardhave)
  have their blocks rejected and not only lose the reward, but also waste the
  effort expended to find a Proof-of-Work solution.
  \item Independent selection, by every node, of the chain with the most
  cumulative computation demonstrated through Proof-of-Work
\end{itemize}


\paragraph{The 51\% attack} This consensus mechanism is vulnerable to the
so-called 51\% attack, which can be carried out by a group of miners controlling
more than 50\% of the total network hashing power. In this situation the
attackers would be able to prevent new transactions from gaining confirmations,
allowing them to halt payments between some or all users. The attackers would
also be able to reverse transactions that were completed while they were in
control of the network, meaning they could double-spend coins. This attack is
however hypothetical in Bitcoin and even if it was carried out the attacker
wouldn't be able to create new coins or alter old blocks.






















\begin{figure}[b]
	\centering
	\includegraphics[width=1\linewidth]{img/transaction-script-example.png}
	\caption{Example of a Script program. Image taken from reference \cite{antonopoulos2017mastering}}
	\label{fig:script-example}
\end{figure}

  \section{Bitcoin anonymity issues} In bitcoin the transactions are exchanged
between addresses, which, as explained in the previous chapters, are basically
hashes of public keys. The purpose of these addresses is to serve as pseudonyms
and provide some anonymity. However, since all Bitcoin transactions are stored
in a publicly available ledger and they basically consist of a chain of
digital signatures which provide cryptographic proofs of funds transfer, Bitcoin
privacy concerns were raised and in the last few years, researchers have shown
that Bitcoin anonymity is much weaker than was initially expected. Users’
transactions in fact can often be easily linked together and if anyone of those
transactions is linked to the user’s identity, then all of its transactions may
be exposed. For these reasons Bitcoin is sometimes compared to a bank making
bank statements publicly available online but blanking out the names.

\paragraph{Tainted bitcoins} The main consequence of this lack of anonymity is
that the history of each bitcoin can be traced and therefore some funds (for example
stolen bitcoins or bitcoins known to have been used for illegal purposes) can be
marked as \emph{tainted}, deflating consequently their value.  This can be done
for example by warning users to don't accept coins that come from a given
address using alert messages or by coding a list of banned Bitcoin addresses
within the official Bitcoin client releases. Also, users are less likely to
accept coins that are tainted since owning them  can be a risk and they can ask
the payer to use different coins as payment.


\paragraph{Clients privacy measures} Besides addresses, Bitcoin clients
adopt some more privacy measures. These measures consist of allowing users to
have more than one address and encouraging them to frequently change their
addresses by transferring some of their bitcoins to the newly created addresses.
Moreover, for each user, a new address is automatically created and used for
collecting the change resulting from any transaction of the user. These
addresses are called \emph{shadow addresses}.







%%%%%%%%%%%%%%%%%%%%%%%%%%%
% *** FIRST SUB-SECTION ***
%%%%%%%%%%%%%%%%%%%%%%%%%%%
\subsection{Compromise of privacy examples} In the following section, it will be
shown some examples of how user privacy can be compromised by exploiting the
existing Bitcoin client implementations and carrying out a behavior-based
analysis of the public ledger \cite{karame2016bitcoin}. It's important to point
out that there are also other kinds of attacks which operates at the network
layer and which allow the attacker to obtain user information from the Bitcoin
peer to peer network \cite{karame2016bitcoin}, which however will not be
discussed in this book.

\paragraph{Exploiting multi input transactions} The first method for obtaining
users information consists of observing the multi-input transactions. A
multi-input transaction, as discussed in chapter \ref{sec:transactions}, is a
transaction which accepts more UXTO as input. If in a transaction these UXTOs
are owned by different addresses, then it is straightforward to conclude that
the input addresses belong to the same user.

\paragraph{Exploiting shadow addresses} Another method is to exploit the shadow
addresses generated by the Bitcoin clients. When a Bitcoin transaction has $n$
output addresses $\{a_1 , \dots, a_n \}$ (transaction with multiple recipients)
and only one address is a new address (namely the address has never appeared in
the ledger before) it is then possible to assume that the newly appearing
address is a shadow address for the user that sent the transaction.

\paragraph{Behavior-based analysis} Besides exploiting Bitcoin client
implementations, an attacker can also use behavior-based clustering algorithms
like K-Means (KMC) and Hierarchical Agglomerative Clustering (HAC) for profiling
Bitcoin users. Without going into the details of these techniques, in reference
\ref{} these algorithms has been tested in a simulated Bitcoin system and the
achieved results shows that given 200 simulated user profiles, almost 42\% of
the users have their profiles captured with 80\% accuracy and the profile
leakage in Bitcoin is larger when users participate in a large number of
transactions, while decreases as the number of transactions performed by the
user decreases.

  
\section{Enhancing Bitcoin privacy}
There are basically two approaches for enhancing users privacy in Bitcoin:
\begin{itemize}
  \item Mixing services: they achieve users privacy generally without degrading
  the performance of the system. However, they require absolute trust in a third
  party.
  \item Cryptograpic extensions of Bitcoin: extensions of the Bitcoin protocol
  which eliminate the need for trusted third parties but tend to be less efficient
  in terms of performance.
\end{itemize}


%%%%%%%%%%%%%%%%%%%%%%%%%%%
% *** FIRST SUB-SECTION ***
%%%%%%%%%%%%%%%%%%%%%%%%%%%
\subsection{Mixing services} A bitcoin mixing service act as mediators and
provides anonymity by transferring payments from an input set of bitcoin
addresses to an output set of bitcoin addresses, such that is it hard to trace
which input address paid which output address, as schematized in figure
\ref{fig:mixing-service-scheme}. Examples of this kind of mixing services are
Mixcoin and CoinParty. The former relies on a third party can violate users
privacy and steal users’ bitcoins (theft is detected but not prevented), while
the second uses more mixing parties and it is considered secure only if $2/3$ of
the mixing parties are honest.

There is also another kind of mixing services in which the service acts as a
``coin hisory resetter''. In this case the user sends to the mixer a certain
amount of bitcoin and a return address and the mixer sends back to the user (to
the specified address) someone else’s coins of the same value. Examples of this
kind of services are BitLaundry and Bitcoin Fog. The problem of these services is
that they do not protect form network-layer attacks since the eventually the
user is the one making payments (instead of the mixing service).

\begin{figure}[!htb]
	\centering
	\includegraphics[width=1\linewidth]{img/mixing-service-scheme.png}
	\caption{Scheme of how a mixing service works}
	\label{fig:mixing-service-scheme}
\end{figure}






%%%%%%%%%%%%%%%%%%%%%%%%%%%
% *** SECOND SUB-SECTION ***
%%%%%%%%%%%%%%%%%%%%%%%%%%%
\subsection{Enhancing privacy through blind signatures}\label{sec:enh-sign} In
the following section is presented a mixing service scheme for enhancing Bitcoin
privacy through the use of blind signatures. The scheme has been proposed  by E.
Heilman, F. Baldimtsi and S. Goldberg \cite{heilman-blindly-signed-contracts},
is based on the scheme used in eCash \cite{Chaum1984} and, unlike other previous
schemes that are efficient but achieve limited security/anonymity or other which
provide strong anonymity but are slow and require large numbers of transactions,
it provides anonimity at reasonable speed using an untrusted third party (which
can therefore be malicious).

\subsubsection{High-level overview of the scheme}
\paragraph{Scenario}
The scenario is the following: $A$, \emph{the payer}, wants to anonymously send
1 bitcoin to $B$, \emph{the  payee}. If $A$ performed a standard transaction
sending 1 BTC from $address_A$ (owned by $A$) to a fresh ephemeral address
$address_B$ (owned by $B$), there would be a record in the blockchain linking
the two addresses. Even if $A$ and $B$ always create a fresh address for each
payment they receive, the links between addresses can be used to de-anonymize
users when they for example have a transaction with a third party which learns
their identify (e.g. their email address). The basic idea is to used a third
party $I$ that breaks the link between $A$ and $B$ addresses: $A$ sends coins to
$I$ and $I$ sends different coins for the same value to $B$, acting thus as a
mixing service. If other users use $I$ and enough transactions pass through it,
it becomes difficult for an attacker to link $A$ and $B$.

\emph{The main problem is that $I$ knows everything about the transactions between
$A$ and $B$}.

\paragraph{eCash scheme}
A possible solution to this issue is the scheme used in eCash for preventing $I$
from knowing who $A$ wants to pay. This scheme is shown in figure
\ref{fig:eCash-scheme} and relies on blind signatures. $A$ chooses a random
serial number $sn$, blinds it to $\overline{sn}$ and asks $I$ to compute a blind
signature $\overline\sigma$ on $\overline{sn}$, which sends back to $A$. $A$
unblinds these values to obtain $V = (sn, \sigma)$ and then pays $B$ using the
voucher $V$. Finally, $B$ redeems $V$ with $I$ to obtain the bitcoin. With this
scheme $I$ does not know who $A$ wants to pay it cannot read the blinded serial
number $\overline{sn}$ that it signs and it cannot link a message/signature
$(sn, \sigma)$ pair to its blinded value $(\overline{sn}, \overline\sigma)$.
Blindness therefore ensures that $I$ cannot link a voucher it redeems with a
voucher it issues. Blind signatures are also unforgeable, which ensures that a
malicious user cannot issue a valid voucher to itself.
\begin{figure}[!htb]
	\centering
	\includegraphics[width=0.6\linewidth]{img/eCash-scheme.png}
	\caption{eCash protocol scheme}
	\label{fig:eCash-scheme}
\end{figure}

\paragraph{Heilman scheme} The main problem with the eCash approach is that $I$
has to be honest: if $I$ is malicious it could refuse to issue a voucher to $A$
after receiving its bitcoin and thus the scheme fails. To solve this, the scheme
proposed in \cite{heilman-blindly-signed-contracts} uses Bitcoin
\emph{transaction contracts} for achieving blockchain-enforced \emph{fair
exchange}. An high level view of the scheme is shown in figure
\ref{fig:heilman-scheme}. The scheme consists of four blockchain transactions
that are confirmed in three blocks on the blockchain and the key idea is that
$A$ transfers a bitcoin to $I$ if and only if it receives a valid voucher $V$ in
return. The four transactions implement two fair exchanges:
\begin{itemize}
  \item $V\rightarrow BTC$, which consists of the transactions (1) $T_{offer(V\rightarrow
  BTC)}$ and (2) $T_{fulfill(V\rightarrow BTC)}$ and it ensures that a malicious $I$
  cannot redeem $B$’s voucher without providing $B$ with a bitcoin in return.
  The exchange stands in for the interaction between $B$ and $I$.
  Transaction (1) offers a fair exchange of one bitcoin (from $I$) for
  one voucher (from $B$), while transaction (2) is created by $B$ to meet
  the offer by $I$.
  \item $BTC\rightarrow V$, which consists of the two transactions (1) $T_{offer(BTC\rightarrow
  V)}$ and (2) $T_{fulfill(BTC\rightarrow V)}$ and it ensures that a malicious $I$
  cannot take a bitcoin from $A$ without providing it with a voucher $V$.
\end{itemize}


\begin{figure}[!htb]
	\centering
	\includegraphics[width=0.6\linewidth]{img/heilman-scheme.png}
	\caption{High-level view of the scheme proposed in \cite{heilman-blindly-signed-contracts}}
	\label{fig:heilman-scheme}
\end{figure}

\subsubsection{Fair exchange implementation} In the following section it will be
explained how the transaction contracts $T_{offer(BTC\rightarrow V)}$ and (2)
$T_{fulfill(BTC\rightarrow V)}$ implement the fair exchange $BTC\rightarrow V$
used in our protocol scheme shown in figure \ref{fig:heilman-scheme}. The
implementation for the exchange $V\rightarrow BTC$ is analogous.

As already mentioned, the fair exchanges are achieved using transaction
contracts. A transaction contract can be implemented using \emph{Script}, a
simple programming language provided by Bitcoin which, as discussed in chapter
\ref{sec:scripts}, allows to associate each transaction with a script which
defines the rules for spending the transaction outputs, namely how the next
person wanting to spend the bitcoins being transferred can gain access to them.

In this case, the CHECKLOCKTIMEVERIFY feature of Script is used in order to
timelock a transaction, so that funds can be reclaimed if a contract has not
been spent within a given time window $tw$. For implementing the $BTC\rightarrow
V$ fair exchange, $A$ generates the transaction contract
$T_{offer(BTC\rightarrow V)}$ which says the following:

``\emph{$A$ offers bitcoins to $I$ under the condition that $I$ must compute a
valid blind signature on the blinded serial number $\overline{sn}$ (provided by $A$)
within time window $tw$. If this condition is not satisfied, the
bitcoin reverts to $A$.}''

The output of this transaction therefore can be spent in a future transaction $T_f$
only if one of the following conditions is met:
\begin{enumerate}
  \item $T_f$ is signed by $I$ and contains a valid blind signature $\overline\sigma$ on $\overline{sn}$
  \item $T_f$ is signed by $A$ and the time window $tw$ has expired.
\end{enumerate}
For fulfilling the contract and acquiring the bitcoins of the transaction $I$
has therefore to satisfy the condition $2$, which means that $I$ has to post a
transaction $T_{fulfill(BTC\rightarrow V)}$ that contains a valid blind
signature $\overline\sigma$ on $\overline{sn}$. If $I$ does not fulfill the
contract within the time window $tw$, then the condition $2$ is met when $A$
signs and posts a transaction $T_f$ that returns back the offered bitcoins.

\subsubsection{Blind signature scheme} The scheme used for the blind signature
is the Boldyreva’s scheme \cite{boldyreva2003threshold}, which requires two
rounds of interaction. Since the elliptic curve defined by the standard
Secp256k1 used by Bitcoin doesn't support the bilinear pairings required for the
adopted signature scheme, for using the scheme it is necessary to slightly
modify the Bicoin protocol in order to adopt a different elliptic curve.

Let $\mathbb{G}$ be a cyclic additive group of order $p$ (with $p$ prime) in
which the Diffie-Hellman problem is hard and $\mathbb{G'}$ a cyclic
multiplicative group of prime order $q$. Let $e\colon
\mathbb{G}\times\mathbb{G}\to\mathbb{G'}$ be the bilinear pairing, $g$ be a
generator of the group $\mathbb{G}$ and $H$ be a hash function mapping arbitrary
strings to elements of $\mathbb{G}\setminus \{1\}$. The public parameters are
$(p, g, H)$ while the signer public/private key pair is $(sk , pk = g^{sk})$.
The signature scheme works as follow:
\begin{itemize}
  \item To blind $sn$, user $A$ picks random $r\in \mathbb{Z}^{*}_p$ and sets
  $\overline{sn}=H(sn)g^r$.
  \item To sign $\overline{sn}$, signer $I$ computes $\sigma = \overline{sn}^{sk}$.
  \item To unblind the blind signature $\overline{\sigma}$, user $A$ computes
  $\sigma = \overline{\sigma} {pk}^{\-- r}$.
  \item To verify the signature $\sigma$ on $sn$, anyone holding $pk$ checks
  that the bilinear pairing $e(pk,H(sn))$ is equal to $e(g,\sigma)$.
  For verifying the blinded signature $\overline\sigma$ on the blinded
  $\overline{sn}$, anyone holding $pk$ checks if $e(pk,\overline m)= e(g,\overline\sigma)$.
\end{itemize}


\subsubsection{Anonymity considerations} While in the eCash protocol of figure
\ref{fig:eCash-scheme} the anonymity level of users depends on the total number
of payments using $I$, in the scheme proposed by Heilman shown in figure \ref{fig:heilman-scheme} the anonimity level
depends on the number of payment through $I$ in a given epoch. The protocol in
fact runs in epochs and provides set-anonymity within each epoch. An epoch is the
three-blocks window in which the four transaction required by the protocol are
confirmed and stored, as shown in figure \ref{fig:epochs}.
\begin{figure}[!htb]
	\centering
	\includegraphics[width=0.6\linewidth]{img/epochs.png}
	\caption{}
	\label{fig:epochs}
\end{figure}

The anonimity considerations about the proposed scheme are based on the following
assumptions:
\begin{itemize}
  \item all the users coordinate on epoch so that the transactions arrangement
  shown in figure \ref{fig:epochs} is respected (e.g. they choose the starting
  block so that its height\footnote{height=distance from the genesis block}
  is multiple of three)
  \item the payer $A$ and the payee $B$ trust each other. If $A$ or $B$ were malicious
  they could easily conspire with $I$ revealing the other part identity: for exmple
  $A$ could comunicate $I$ the serial number of the received voucher so that when
  $I$ can identify $B$ when he redeems that voucher
  \item payees B always receive payments in a fresh ephemeral address $address_B$
  \item payers only make one anonymous payment per epoch. Similarly, payees only
  accept one payment per epoch.
\end{itemize}

\paragraph{Epoch set-anonimity} As consequence of assumptions (2) and (3), in
every epoch there are exactly $n$ addresses making payments (playing the role of
payer $A$) and $n$ receiving addresses (playing the role of $B$). Anyone looking
at the blockchain can therefore see the participating addresses of payers and
payees, but the probability of successfully linking any chosen payer $A$ to a
payee $B$ should not be more than $1/n$, namely an attacker observing the
blockchain can do no better than randomly guessing who paid whom during an epoch.

\paragraph{Transparency of anonimity} Users in learn the size of their anonymity
set (number of partecipants in their epoch) only after a transaction completes
by looking at the blockchain.  If particular $B$ feels his anonymity set is too
small in one epoch, he can increase the size of it by using the scheme as  a
mixing service and making a new transaction to another address owned by him. For
example:  $address_B$ gets paid in an epoch with $n = 4$. $B$ can create a fresh
ephemeral address $address'_B$ and have $address_B$ pay $address'_B$ in a
subsequent epoch. If the subsequent epoch has a $n = 100$, then B increases the
size of his anonymity set.

\paragraph{Intersection attack} As pointed out by the autors in
\cite{heilman-blindly-signed-contracts}, there's the possibility for an attacker
(or anyone looking at the blockchain) to attempt an intersection attack that
de-anonymaze users across different epochs by using frequencial analysis.

  \section{Bitcoin and Blockchain scalability}
\subsection{Introduction}
As a consequence of the increasing adoption of Blockchain-based cryptocurrencies,
their ability to scale has raised concerns and has received a lot of attention
in the last few years. In particular, the key concerns are:
\begin{itemize}
  \item[-] can cryptocurrencies based on decentralized blockchains be scaled up
  to match the performance of a mainstream payment processor?
  \item[-] what does it take to get there?
\end{itemize}

\paragraph{Bitcoin current performance} As reference, Bitcoin today requires
around 10 minutes to confirm a transaction (a new block is mined every $\sim10$
minutes) and achieves a maximum throughput of 7 transaction/sec
\cite{wikipedia_scalability_2018}. Since the transactions are confirmed only
after the block they belong to is created and added to the blockchain, the
maximum throughput of Bitcoin is effectively capped at maximum block size
divided by block interval. In comparison, a payment processor such as Visa
credit card processes 2000 transactions/sec on average, with a peak rate of
56,000 transactions/sec \cite{wikipedia_scalability_2018}.

\paragraph{Block size debate} A solution for increasing Bitcoin throughput could
therefore be to increase the size of the block, which is currently 1MB, in order
to increase the amount of transactions confirmed every 10 minutes. In the last
few years there's been a debate about this topic which splitted the community.
People in favour for increasing the size claim that increasing it would allow
Bitcoin to easlily reach VISA (and anolog payment systems) numbers, while the
opposing ones claim that this would damage decentralization because blocks of
big size require a lot of computational power for being mined and this increases
the costs of participation, centralizing the miners in a few powerfull nodes.
Their proposed solutions therefore consist of spending effort for optimizing
the use of the current block space available and offloading certain processing
to off-chain networks (off-chain solutions).


  \printbibliography

\end{document}
