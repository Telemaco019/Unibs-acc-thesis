\section{Bitcoin}\label{sec:Bitcoin}
%%%%%%%%%%%%%%%%%%%%%%%%%%%
% *** FIRST SUB-SECTION ***
%%%%%%%%%%%%%%%%%%%%%%%%%%%
\subsection{Introduction}

Bitcoin it's the first fully decentralized cryptocurrency. It was invented by
Satoshi Nakamoto in 2008 and it was the first real implementation of Blockchain.
Bitcoin can be either defined as a protocol, a digital currency and a platform.

Bitcoin can be seen as a combination of
\vspace{-\topsep}
\begin{itemize}
  \item[-] a decentralized peer-to-peer-network (the Bitcoin protocol)
  \item[-] a public transaction ledger (the blockchain)
  \item[-] a set of rules for validating transactions (consensus rules)
  \item[-] a mechanism for reaching distributed consensus on the blockchain (distributed
  consensus algorithm)
\end{itemize}
\vspace{-\topsep}
that allows the usage of the digital currency named bitcoin.

From now on, Bitcoin with the capital $B$ will refer to the Bitcoin protocol while bitcoin with the lowercase
$b$ will refer to the bitcoin currency.



Bitcoin is a distributed peer-to-peer system in which users can exchange
currency over the network just as it can be done with conventional currency.
However, unlike traditional currencies, bitcoins are enterely virtual and thus
there are no physical coins. In particular, there are not even virtual coins since
they are implied in the transactions that send value from a sender to a receiver:
users have private keys which allow them to prove the ownership of bitcoins and
sign transactions in order to unlock the value and transfer it to another user.
These keys are the only requirement for spending bitcoins and therefore they are
protected in wallets stored in the user's devices.
















%%%%%%%%%%%%%%%%%%%%%%%%%%%
% *** SECOND SUB-SECTION ***
%%%%%%%%%%%%%%%%%%%%%%%%%%%
\subsection{Bitcoin protocol specification}
\subsubsection{The reference implementation}
Bitcoin is an open source project and is developed by a community of volunteers.
The first implementation was released by Satoshi Nakamato in 2008 (the only member
of the development community at the time). That implementation during the years
has been heavily modified and improved evolving into what is known as \emph{Bitcoin
Core}, which is now the reference implementation of the Bitcoin system. This
implementation is considered the authoritative one and it specifies how each
part of the system has to be implemented.


\subsubsection{Keys and Addresses}
As mentioned in this chapter introduction, ownership of bitcoin is established
through digital keys, bitcoin addresses, and digital signatures.

In order to be included in the Bitcoin blockchain, transactions require a valid
signature which can be generated only with a private (secret) key. The private
key therefore proves the ownership of bitcoins by signing transactions and
transferrig value from a user to another. Keys come in pairs consisting of a
private (secret) key and a public key and they are generated through Elliptic
Curve Cryptography. In analogy with the traditional banking,the public key can be
seen as the bank account number while the private key as the secret PIN (or the
signature on a check) which provides control over the account by allowing to unlock
the value and transferring it to other people.

\paragraph{Adresses}
An address is unique string of digits and characters which identify the originator
and/or the destination of a transaction. Addresses are derived from public keys
through one-way cryptographic hashing in order to obtain the public key fingerprint.
In particular, a Bitcoin address is derived by hashing the user's public key it twice,
first with the SHA-256 algorithm and then with RIPEMD160. This produces a 160-bit
hash, which is then prefixed with a version number and finally encoded using
Base58Check encoding. The final result is a 26-35 chatacters string which begins
with 1 (public key address) or 3 (pay-to-script-hash address) and it looks like the
the string below:
\begin{center}
  \texttt{1J7mdg5rbQyUHENYdx39WVWK7fsLpEoXZy}
\end{center}

\paragraph{Base58 and Base58Check}
Base58 is an encoding scheme which allows to represent long numbers as alphanumeric
strings. It is a subset of Base64, which represent numbers using 26 lowercase
letters, 26 capital letters, 10 numerals, and 2 more ``special'' characters and
it's usually used to encode email attachments. In particular, Base58 is Base64
without all that characters that are frequently mistaken for one another, namely
it is Base64 without the 0 (number zero), O (capital o), l (lower L), I (capital i)
and the two special characters. Base58Check} is a Base58 encoding with
an additional checksum of four bytes added to the end of the data that is being
encoded which prevents a mistyped bitcoin address from being accepted by the wallet
software as a valid destination.
