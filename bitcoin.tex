\section{Bitcoin}\label{sec:Bitcoin}
%%%%%%%%%%%%%%%%%%%%%%%%%%%
% *** FIRST SUB-SECTION ***
%%%%%%%%%%%%%%%%%%%%%%%%%%%
\subsection{Introduction}

Bitcoin it's the first fully decentralized cryptocurrency. It was invented by
Satoshi Nakamoto in 2008 and it was the first real implementation of Blockchain.
Bitcoin can be either defined as a protocol, a digital currency and a platform.

Bitcoin can be seen as a combination of
\vspace{-\topsep}
\begin{itemize}
  \item[-] a decentralized peer-to-peer-network (the Bitcoin protocol)
  \item[-] a public transaction ledger (the blockchain)
  \item[-] a set of rules for validating transactions (consensus rules)
  \item[-] a mechanism for reaching distributed consensus on the blockchain (distributed
  consensus algorithm)
\end{itemize}
\vspace{-\topsep}
that allows the usage of the digital currency named bitcoin.

From now on, Bitcoin with the capital $B$ will refer to the Bitcoin protocol while bitcoin with the lowercase
$b$ will refer to the bitcoin currency.



Bitcoin is a distributed peer-to-peer system in which users can exchange
currency over the network just as it can be done with conventional currency.
However, unlike traditional currencies, bitcoins are enterely virtual and thus
there are no physical coins. In particular, there are not even virtual coins since
they are implied in the transactions that send value from a sender to a receiver:
users have private keys which allow them to prove the ownership of bitcoins and
sign transactions in order to unlock the value and transferring it to another user.
These keys are the only requirement for spending bitcoins and therefore they are
protected in wallets stored in the user's devices.
















%%%%%%%%%%%%%%%%%%%%%%%%%%%
% *** SECOND SUB-SECTION ***
%%%%%%%%%%%%%%%%%%%%%%%%%%%
\subsection{Bitcoin protocol specification}
