\section{Bitcoin and Blockchain scalability}
\subsection{Introduction}
As a consequence of the increasing adoption of Blockchain-based cryptocurrencies,
their ability to scale has raised concerns and has received a lot of attention
in the last few years. In particular, the key concerns are:
\begin{itemize}
  \item[-] can cryptocurrencies based on decentralized blockchains be scaled up
  to match the performance of a mainstream payment processor?
  \item[-] what does it take to get there?
\end{itemize}

\paragraph{Bitcoin current performance} As reference, Bitcoin today requires
around 10 minutes to confirm a transaction (a new block is mined every $\sim10$
minutes) and achieves a maximum throughput of 7 transaction/sec
\cite{wikipedia_scalability_2018}. Since the transactions are confirmed only
after the block they belong to is created and added to the blockchain, the
maximum throughput of Bitcoin is effectively capped at maximum block size
divided by block interval. In comparison, a payment processor such as Visa
credit card processes 2000 transactions/sec on average, with a peak rate of
56,000 transactions/sec \cite{wikipedia_scalability_2018}.

\paragraph{Block size debate} A solution for increasing Bitcoin throughput could
therefore be to increase the size of the block, which is currently 1MB, in order
to increase the amount of transactions confirmed every 10 minutes. In the last
few years there's been a debate about this topic which splitted the community.
People in favour for increasing the size claim that increasing it would allow
Bitcoin to easlily reach VISA (and anolog payment systems) numbers, while the
opposing ones claim that this would damage decentralization because blocks of
big size require a lot of computational power for being mined and this increases
the costs of participation, centralizing the miners in a few powerfull nodes.
Their proposed solutions therefore consist of spending effort for optimizing
the use of the current block space available and offloading certain processing
to off-chain networks (off-chain solutions).
