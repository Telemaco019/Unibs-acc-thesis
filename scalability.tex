\section{Bitcoin and Blockchain scalability}
\subsection{Introduction}
As a consequence of the increasing adoption of Blockchain-based cryptocurrencies,
their ability to scale has raised concerns and has received a lot of attention
in the last few years. In particular, the key concerns are:
\begin{itemize}
  \item[-] can cryptocurrencies based on decentralized blockchains be scaled up
  to match the performance of a mainstream payment processor?
  \item[-] what does it take to get there?
\end{itemize}

\paragraph{Bitcoin current performance} As reference, Bitcoin today requires
around 10 minutes to confirm a transaction (a new block is mined every $\sim10$
minutes) and achieves a maximum throughput of 7 transaction/sec
\cite{wikipedia_scalability_2018}. Since the transactions are confirmed only
after the block they belong to is created and added to the blockchain, the
maximum throughput of Bitcoin is effectively capped at maximum block size
divided by block interval. In comparison, a payment processor such as Visa
credit card processes 2000 transactions/sec on average, with a peak rate of
56,000 transactions/sec \cite{wikipedia_scalability_2018}.

\paragraph{Bitcoin reparametrization} A solution for increasing Bitcoin
throughput could therefore changing the block interval time (currently 10
minutes) and increasing upper limit of the block size, which currently is 1MB.
In the last few years there's been a debate about this topic which splitted the
community. People in favour for increasing the size claim that increasing it
would allow Bitcoin to easlily reach VISA (and anolog payment systems) numbers,
while the opposing ones claim that this would damage decentralization because
blocks of big size require a lot of computational power for being mined and this
increases the costs of participation, centralizing the miners in a few powerfull
nodes. Their proposed solutions therefore consist of spending effort for
optimizing the use of the current block space available and offloading certain
processing to off-chain networks (off-chain solutions).

As discussed in \cite{croman-scaling-blockchain}, since scalability is not a
singular metric and it includes various performance and security metrics,
reparametrization can achieve only limited benefits considering the network
performance given by Bitcoin’s current peer-to-peer network protocol and the
willing to maintain its current degree of decentralization. However, it is still
an open question wheter reparametrization alone can address the growth of
Bitcoin to the same order of magnitude of systems like the previously mentioned
VISA. Following the considerations discuessed in
\cite{croman-scaling-blockchain}, the next section will explore the
reparametrization limitations which shows that likely the scaling problem of
Bitcoin (and, more in general, of Blockchain systems) cannot be faced with
reparametrization alone.

\subsection{Reparametrization limitations}

\subsubsection{Current Bitcoin metrics}
\paragraph{Maximum throughput}  Today Bitcoin’s maximumt hroughput is 7
transactions/sec \cite{wikipedia_scalability_2018}. This number is limited by
the maximum block size and the interblock time.

\paragraph{Maximum and average block size} The maximum block size is 1 MB, while
the average size today is around 800 KB \cite{current-block-size}. In the years
2014-2015, when the measurements in \cite{croman-scaling-blockchain} were done,
the average block size was 540 KB.

\paragraph{Latency} The latency is the time required for a transaction obtain a
single confirmation (therefore the time required for the transaction to be
included in a block). It is roughly 10 minutes.





\subsubsection{Metric definitions}
\begin{definition}
  We define the metric ``X\% block propagation delay'' as the time required for
  X\% of the nodes to receive a full block
\end{definition}
\begin{remark}
  The calculation of this metric is done empirically, carrying out measurements
  as done in the experiment conducted in \cite{croman-scaling-blockchain}.
\end{remark}

\begin{definition}\label{def:effective-throughput}
  We define the metric ``X\% effective throughput'' as the block size divided by
  the metric ``X\% block propagation delay'':

  \[\text{X\% effective throughput} = \frac{\text{block size}}{\text{X\% block
  propagation delay}}\]
\end{definition}

\begin{remark}\label{remark-1}
  If the transactions rate (or, equivalently, the average block size) exceeds
  the X\% effective throughput during a block interval, then $(100-X)\%$
  of the nodes on the network would be unable to receive blocks as they arrive,
  so they would be effectively disabled. Table \ref{tab:effective-throughput-values}
  shows the connection between transactions rate and X\% effective throughput.
\end{remark}

\begin{remark}
  If the size or properties of the Bitcoin network change then the X\% effective
  throughput also changes.
\end{remark}

\begin{table}[h!]
\centering
\resizebox{\textwidth}{!} & \textbf{Effective throughput} & \textbf{Transactions/sec}                                    \\ \midrule
50\% & 492 Kbps & 248 trans./sec \\
\\
90\% & 55 Kbps & 26 trans./sec \\ \bottomrule
\end{tabularx}
}
\caption{Effective throughtputs and associated transactions per second. The values
are calculated assuming that 50\% and 90\% block propagation times are 8.7 seconds
and 79 seconds respectively, while the average block size is 540KB and the
transactions are 250-byte transactions. Note that the effective throughput
values are expressed in kilo-bits per second.}
\label{tab:effective-throughput-values}
\end{table}




%Interesting point offerend by article ***: their conclusion is that fundamental
%protocol redesign is needed for blockchains to scale significantly while
%retaining their decentralization (reparametrization only is not enough, as
%explained in SECTION 3). They also discuss about new strategies for designing
%new protocols by addressing blockchain limitations through a partition of the
%system in different layers, analyzing the bottlenecks and the limitations of
%each layer (SECTION 4).

\subsubsection{Throughput and latency limits}

\paragraph{Premises} It is assumed that it is desired to maintain nearly the
current level of decentralization in the system, namely it is desired to have at
least 90\% of properly functioning nodes. It is also assumed that the current
Bitcoin peer-to-peer network is used, without apporting any changes to it since
they would affect the X\% effective throughput.

\paragraph{Throughput limit} given the two premises above, given the definition
\ref{def:effective-throughput} and given the remark \ref{remark-1}, the block size and
interval must satisfy:
\[ \frac{\text{block size}}{\text{X\% effective throughput}} < \text{block interval} \]
\emph{Consequence: given the current overlay network and today’s 10 min average
block interval, the block size should not exceed 4 MB \cite{croman-scaling-blockchain}.
A 4 MB block size corresponds to a throughput of at most 27 transactions/sec
($\text{maximum throughput} = \text{block size} / \text{block interval}$)}.


\paragraph{Latency limit} For improving the system latency a simple idea is to
reduce the block interval. In order to do so and maintain an high effective
throughput is however necessary to reduce the block size (consequence of the
throughput limit formula shown in the previous paragraph), but using a block too
small would not make full use of the network bandwidth. The following guideline
is the result of the experiments and measurements carried out in
\cite{croman-scaling-blockchain}:  \emph{given today’s overlay network, to retain at
least 90\% effective throughput and fully utilize the bandwidth of the network,
the block interval should not be significantly smaller than 12s.}





\subsection{Bitcoin bottlenecks} In this section will be discussed techniques
alternative to the reparametrization that allow the Bitcoin system to scale. In
particular, the system will be decomposed into a set of abstraction layers
called \emph{planes} (network, consensus, storage, view and side planes), which
allows limitations and bottlenecks to be addressed at each plane individually
and in a structured manner.
