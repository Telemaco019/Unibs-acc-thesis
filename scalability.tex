\section{Bitcoin and Blockchain scalability}
\subsection{Introduction}
As a consequence of the increasing adoption of Blockchain-based cryptocurrencies,
their ability to scale has raised concerns and has received a lot of attention
in the last few years. In particular, the key concerns are:
\begin{itemize}
  \item[-] can cryptocurrencies based on decentralized blockchains be scaled up
  to match the performance of a mainstream payment processor?
  \item[-] what does it take to get there?
\end{itemize}

\paragraph{Bitcoin current performance} As reference, Bitcoin today requires
around 10 minutes to confirm a transaction (a new block is mined every $\sim10$
minutes) and achieves a maximum throughput of 7 transaction/sec
\cite{wikipedia_scalability_2018}. Since the transactions are confirmed only
after the block they belong to is created and added to the blockchain, the
maximum throughput of Bitcoin is effectively capped at maximum block size
divided by block interval. In comparison, a payment processor such as Visa
credit card processes 2000 transactions/sec on average, with a peak rate of
56,000 transactions/sec \cite{wikipedia_scalability_2018}.

\paragraph{Bitcoin reparametrization} A solution for increasing Bitcoin
throughput could therefore be to change the block interval time and to increase
the size of the block, which is currently 1MB, in order to increase the amount
of transactions confirmed every 10 minutes. In the last few years there's been a
debate about this topic which splitted the community. People in favour for
increasing the size claim that increasing it would allow Bitcoin to easlily
reach VISA (and anolog payment systems) numbers, while the opposing ones claim
that this would damage decentralization because blocks of big size require a lot
of computational power for being mined and this increases the costs of
participation, centralizing the miners in a few powerfull nodes. Their proposed
solutions therefore consist of spending effort for optimizing the use of the
current block space available and offloading certain processing to off-chain
networks (off-chain solutions).

As discussed in \cite{}, since scalability is not a singular metric and it
includes various performance and security metrics, reparametrization can achieve
only limited benefits considering the network performance given by Bitcoin’s
current peer-to-peer network protocol and the willing to maintain its current
degree of decentralization. However, it is still an open question wheter
reparametrization alone can address the growth of Bitcoin to the same order of
magnitude of systems like the previously mentioned VISA. Following the
considerations discuessed in \cite{}, the next section will explore the
reparametrization limitations which shows that likely the scaling problem of
Bitcoin (and, more in general, of Blockchain systems) cannot be faced with
reparametrization alone.


Interesting point offerend by article ***: their conclusion is that fundamental
protocol redesign is needed for blockchains to scale significantly while
retaining their decentralization (reparametrization only is not enough, as
explained in SECTION 3). They also discuss about new strategies for designing
new protocols by addressing blockchain limitations through a partition of the
system in different layers, analyzing the bottlenecks and the limitations of
each layer (SECTION 4).
